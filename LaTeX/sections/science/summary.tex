\begin{frame}{}
    \LARGE Generative AI for Science: \textbf{Summary and Future Directions}
\end{frame}

\begin{frame}[allowframebreaks]{Summary and Future Directions}
    \textbf{Key Takeaways}
    \begin{itemize}
        \item Generative AI is transforming scientific research across various domains.
        \item It enhances the speed and accuracy of simulations, predictions, and data analysis.
        \item Applications range from protein design to climate modeling and drug discovery.
    \end{itemize}

    \framebreak

    \textbf{Future Directions}
    \begin{itemize}
        \item Continued advancements in AI models will lead to even more sophisticated simulations and predictions.
        \item Integration of AI with experimental data will enhance the reliability of scientific findings.
        \item Ethical considerations and responsible AI use will be crucial as these technologies evolve.
    \end{itemize}

    \framebreak

    \textbf{Conclusion}
    \begin{itemize}
        \item Generative AI holds immense potential to accelerate scientific discovery and innovation.
        \item Ongoing research and development will unlock new possibilities in understanding complex systems.
        \item Collaboration between AI experts and domain scientists is essential for maximizing impact.
    \end{itemize}
\end{frame}