\section{Objectives \& Learning Outcomes}
\begin{frame}{}
    \LARGE GANs: \textbf{Objectives \& Learning Outcomes}
\end{frame}

\begin{frame}[allowframebreaks]{GANs: Objectives}
    \begin{enumerate}
        \setlength{\itemsep}{-0.1em}
        \item \textbf{Understand the fundamental concepts and motivations behind GANs.}
        \begin{enumerate}
            \setlength{\itemsep}{-0.75em}
            \item Learn why GANs were introduced and what problems they aim to solve in generative modeling.
        \end{enumerate}
        \item \textbf{Explore the mathematical formulation and training process of GANs.}
        \begin{enumerate}
            \setlength{\itemsep}{-0.75em}
            \item Study the minimax objective and the roles of the generator and discriminator.
            \item Analyze how GANs compare distributions via samples.
        \end{enumerate}
        \item \textbf{Examine objective functions and training challenges.}
        \begin{enumerate}
            \setlength{\itemsep}{-0.75em}
            \item Review standard and alternative objective functions.
            \item Identify common problems in GAN training, such as mode collapse and instability.
        \end{enumerate}
        \item \textbf{Survey popular GAN variants and latent space concepts.}
        \begin{enumerate}
            \setlength{\itemsep}{-0.75em}
            \item Wasserstein GANs, Conditional GANs, CycleGANs, StyleGANs.
            \item Understand the role and selection of latent spaces in GANs.
        \end{enumerate}
    \end{enumerate}
\end{frame}

\begin{frame}[allowframebreaks]{\textcolor{green}{$\Rightarrow$} Learning Outcomes}
    By the end of this session, you will be able to:
    \begin{enumerate}
        \item Understand how GANs work at a high level.
        \item Describe the roles of the Generator and Discriminator.
        \item Explain how GANs are trained.
        \item Recognize major variants of GANs.
        \item Understand challenges and limitations of GANs.
    \end{enumerate}
\end{frame}