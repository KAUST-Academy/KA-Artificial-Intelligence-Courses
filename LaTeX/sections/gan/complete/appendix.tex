\section*{Appendix}
\begin{frame}[allowframebreaks]{Appendix}
\begin{block}{GANs: Definition and Core Components:}
    \begin{itemize}
        \item \textbf{Definition}: Generative Adversarial Networks (GANs) consist of two neural networks trained in opposition to each other.
        \item \textbf{Core Components:}
        \begin{itemize}
            \item \textbf{Generator (G):} Takes random noise (latent vector) and produces fake data samples.
            \item \textbf{Discriminator (D):} Receives real or fake data and predicts whether the input is real.
        \end{itemize}
        \item \textbf{Mathematical Formulation:}
    \end{itemize}
    \begin{equation*}
        \min_G \max_D \; \mathbb{E}_{x \sim p_{\text{data}}} [\log D(x)] + \mathbb{E}_{z \sim p_z} [\log(1 - D(G(z)))]
    \end{equation*}
    \begin{itemize}
        \item[] Where:
        \begin{itemize}
            \item $x$: Real data sample
            \item $z$: Noise vector sampled from prior $p_z$
            \item $G(z)$: Generated sample
        \end{itemize}
    \end{itemize}
\end{block}


\end{frame}