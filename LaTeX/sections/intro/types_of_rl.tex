\section{Types of Reinforcement Learning}
\begin{frame}{}
    \LARGE Reinforcement Learning: \textbf{Types}
\end{frame}

\begin{frame}[allowframebreaks]{Types of Reinforcement Learning}
    \begin{figure}
        \centering
        \fetchconvertimage{https://www.mdpi.com/energies/energies-17-05307/article_deploy/html/images/energies-17-05307-g005.png}{images/intro/types.png}{width=\textwidth,height=0.9\textheight,keepaspectratio}
    \end{figure}
\framebreak
    \textbf{Model-based vs. Model-free}
    \begin{itemize}
        \item \textbf{Model-based:} The agent builds or uses a model of the environment to plan actions.
        \item \textbf{Model-free:} The agent learns to act without explicitly modeling the environment.
    \end{itemize}
    \vspace{1em}
    \textbf{Value-based vs. Policy-based}
    \begin{itemize}
        \item \textbf{Value-based:} The agent learns a value function (e.g., Q-learning) to evaluate actions or states.
        \item \textbf{Policy-based:} The agent directly learns a policy that maps states to actions (e.g., Policy Gradient methods).
    \end{itemize}
\framebreak
    \begin{figure}
        \centering
        \fetchconvertimage{https://lh4.googleusercontent.com/o0SP05CZTDEM7Hn3qais336jpq4pAi_IaTzMvT9NDNdIx3tJd53hx-Tng8QWr9BPMkeen_H84G1a--E2Fqq9D1ArG4djyFhE61FP9xFucCoU-VMDFlhzGmzxqQ54Ejs4QvzdM39plrVHJHNbCmIl2l4}{images/intro/types-compare.png}{width=\textwidth,height=0.9\textheight,keepaspectratio}
    \end{figure}
\framebreak
    \begin{figure}
        \centering
        \fetchconvertimage{https://blog.paperspace.com/content/images/2021/12/image-3.png}{images/intro/types-algo.png}{width=\textwidth,height=0.9\textheight,keepaspectratio}
    \end{figure}
\end{frame}