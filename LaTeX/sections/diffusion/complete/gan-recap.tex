\section{GANs Recap}
\begin{frame}{GANs: Recap}
We have previously discussed Generative Adversarial Networks (GANs), 
which are a powerful class of generative models. 
Here are some key points to remember:
\begin{itemize}
    \item Instead of explicitly modeling the data probability distribution, GANs learn it implicitly.
    \item<2-> Two networks are trained jointly: the \textcolor{purple}{\textbf{generator}} creates fake samples to fool the \textcolor{blue}{\textbf{discriminator}}, while the discriminator tries to distinguish between real and fake samples.
    \item<3-> GANs have been state-of-the-art in image generation due to the high quality of their outputs.
\end{itemize}
\end{frame}
