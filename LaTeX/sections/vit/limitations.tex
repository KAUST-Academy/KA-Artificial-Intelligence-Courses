\begin{frame}[allowframebreaks]{Limitations and Considerations}
    \begin{itemize}
        \item \textbf{Data Requirements:}
        \begin{itemize}
            \item ViTs require large-scale datasets or knowledge distillation to achieve strong performance.
            \item They lack the strong inductive biases of CNNs.
        \end{itemize}

        \item \textbf{Computational Cost:}
        \begin{itemize}
            \item The self-attention mechanism has quadratic complexity with respect to input size, making ViTs computationally expensive.
            \item Techniques such as windowed or masked attention help reduce this cost.
        \end{itemize}

        \item \textbf{Inductive Bias vs. Flexibility:}
        \begin{itemize}
            \item ViTs offer greater flexibility and generality.
            \item They may be less sample-efficient compared to CNNs, which have built-in spatial priors.
        \end{itemize}

        \framebreak

        \item \textbf{Interpretability:}
        \begin{itemize}
            \item Attention maps can provide some interpretability.
            \item They may also highlight irrelevant or noisy regions, limiting their usefulness.
        \end{itemize}

        \item \textbf{Deployment Challenges:}
        \begin{itemize}
            \item ViTs typically require more resources, making them less suitable for low-power or resource-constrained environments.
            \item Research into lightweight variants, pruning, and quantization is ongoing to address this.
        \end{itemize}
    \end{itemize}

    \framebreak

    \begin{itemize}
        \item \textbf{Future Directions:}
        \begin{itemize}
            \item Development of more efficient attention mechanisms to reduce computational demands.
            \item Integration of CNN and transformer architectures to leverage the strengths of both.
            \item Design of lightweight ViTs tailored for edge devices and real-time applications.
        \end{itemize}
    \end{itemize}
\end{frame}