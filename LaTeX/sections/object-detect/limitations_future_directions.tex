\section{Limitations and Future Directions}
\begin{frame}{}
    \LARGE Object Detection: \textbf{Limitations and Future Directions}
\end{frame}

\begin{frame}{Limitations Still to Solve}
    \begin{itemize}
        \item \textbf{Small or Clustered Object Detection}: Current models struggle to accurately detect small or closely packed objects.
        \item \textbf{Robustness in Adverse Conditions}: Performance drops significantly under poor lighting or challenging weather conditions.
        \item \textbf{Dataset Bias and Labeling Cost}: Datasets may not represent real-world diversity, and manual annotation remains expensive.
        \item \textbf{On-device Deployment Complexity}: Running detection models on edge devices is challenging due to resource constraints and licensing issues.
    \end{itemize}
\end{frame}

\begin{frame}{What’s Next? Future Directions}
    \begin{itemize}
        \item \textbf{Transformer-first Models}: Approaches like DETR and RT-DETR are surpassing YOLO in both speed and accuracy.
        \item \textbf{Hybrid Architectures}: Models such as MambaNeXt combine global context modeling with edge efficiency.
        \item \textbf{Weak-supervised \& Prompt-driven Models}: Examples like YOLO-World enable zero-shot detection with minimal supervision.
        \item \textbf{Ongoing Challenges}: Focus areas include improving detection of small/clustered objects, open-vocabulary detection, and developing models suitable for edge deployment.
    \end{itemize}
\end{frame}