\begin{frame}[allowframebreaks]{Anomaly Detection}
    \begin{itemize}
        \item Autoencoders are trained to reconstruct input data.
        \item During training, the model learns to compress and decompress data that is similar to the training set (normal data).
        \item When presented with anomalous (outlier) data, the reconstruction error increases, as the autoencoder cannot effectively reconstruct unseen patterns.
    \end{itemize}

    \textbf{Use Cases:}
    \begin{itemize}
        \item Fraud detection in financial transactions
        \item Fault detection in industrial systems
        \item Intrusion detection in cybersecurity
        \item Medical anomaly detection (e.g., rare diseases in imaging)
    \end{itemize}

    \framebreak
    \begin{figure}
        \centering
        \fetchconvertimage{https://www.mdpi.com/IoT/IoT-04-00016/article_deploy/html/images/IoT-04-00016-g002.png}{images/autoencoders/anomaly-detection.png}{width=0.8\textwidth,keepaspectratio}
        \caption*{Structure of the deep autoencoder (AE) for anomaly detection and feature extraction for multi-classification of cyber-attacks.}
    \end{figure}

    \framebreak
    \begin{figure}
        \centering
        \fetchconvertimage{https://www.researchgate.net/profile/Huaiqiang-Sun/publication/367509994/figure/fig1/AS:11431281115469461@1674921049910/Overview-of-the-proposed-autoencoder-based-anomaly-detection-framework-A-Training-the.png}{images/autoencoders/anomaly-detection-brain.png}{height=0.68\textheight,keepaspectratio}
        \caption*{Autoencoder-based anomaly detection framework: A) Training the autoencoder for modeling the distribution of healthy brain anatomy. B) In the inference phase, anomaly maps reveal lesions by subtracting their reconstruction from the input image. C and D) Details of the proposed autoencoder.}
    \end{figure}
    
\end{frame}