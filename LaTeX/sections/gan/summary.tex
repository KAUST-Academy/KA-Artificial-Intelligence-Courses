\begin{frame}{Summary - GANs}
\begin{itemize}
    \item Instead of explicitly learning data probability distribution. Learn it implicitly.
    \item Train two networks jointly. Generator generates fake samples and aims to fool discriminator. Discriminator aims to discriminate between real and fake samples.
    \item GANs are notoriously difficult to train. Lots of tips and tricks available.
    \item Huge amount of research done on GANs and plenty of variations with interesting results.
\end{itemize}
\end{frame}

\begin{frame}{References}

Reference Slides
\begin{itemize}
    \item Fei-Fei Li "Generative Deep Learning" CS231
    \item Francois Fleuret "Deep Learning" EE559
    \item Murtaza Taj "Deep Learning" CS437
\end{itemize}
    
\end{frame}

\begin{frame}{References (Papers)}

\begin{itemize}
    \item Goodfellow et al., ``Generative Adversarial Networks,'' 2014. \href{https://arxiv.org/abs/1406.2661}{arXiv:1406.2661}
    \item Arjovsky et al., ``Wasserstein GAN,'' 2017. \href{https://arxiv.org/abs/1701.07875}{arXiv:1701.07875}
    \item Zhu et al., ``Unpaired Image-to-Image Translation using Cycle-Consistent Adversarial Networks,'' 2017. \href{https://arxiv.org/abs/1703.10593}{arXiv:1703.10593}
    \item Karras et al., ``A Style-Based Generator Architecture for Generative Adversarial Networks,'' CVPR 2019. \href{https://arxiv.org/abs/1812.04948}{arXiv:1812.04948}
    \item Creswell et al., ``Generative Adversarial Networks: An Overview,'' IEEE Signal Processing Magazine, 2018. \href{https://ieeexplore.ieee.org/document/8280743}{IEEE Xplore}
\end{itemize}

\end{frame}