\begin{frame}{}
    \centering
    \LARGE Autoregressive Models: \textbf{Summary and Limitations}
\end{frame}

\begin{frame}[allowframebreaks]{Summary and Limitations}
    \begin{itemize}
        \item Define a specific order for the variables in the data (e.g., $x_1, x_2, \ldots, x_n$).
        \item Model the joint probability as a product of conditional probabilities: 
        \[
            \mathcal{P}(X = \overline{x}) = \prod_{i=1}^n \mathcal{P}(x_i \mid x_{<i})
        \]
        \item Sampling is performed recursively: generate $x_1$ first, then $x_2$ conditioned on $x_1$, and so on.
        \item Efficient to compute likelihoods and log-likelihoods for both discrete and continuous variables.
        \item Flexible: can be extended to multi-class, multi-dimensional, and structured data.
        \framebreak
        \item Limitations:
        \begin{itemize}
            \item No natural way to extract global features or representations.
            \item Not inherently suited for clustering or unsupervised learning tasks.
            \item The choice of variable ordering can significantly affect performance.
        \end{itemize}
    \end{itemize}
\end{frame}