\section{Motivation}
\begin{frame}{Motivation}
        \begin{itemize}
            \setlength{\itemsep}{1em}
            \item Traditional NLP models are \textbf{unimodal} (text-only) and \textbf{reactive}.
            \item Modern AI agents need to perceive, reason, and act in complex environments using \textbf{multiple modalities} (e.g., vision, text, speech).
            \item \textbf{Multimodal systems} and \textbf{agentic AI} aim to build goal-directed, autonomous, and continuously learning agents.
            \item Rise of agent frameworks (Auto-GPT, BabyAGI) and vision-language models enables new research frontiers in decision-making, robotics, and assistive AI.
        \end{itemize}
\end{frame}

\section{Learning Outcomes}
\begin{frame}{Learning Outcomes}
\begin{itemize}
    \setlength{\itemsep}{1em}
    \item Explain core vision-language models (e.g., CLIP, VisualBERT, FLAVA).
    \item Compare multimodal architectures and their integration mechanisms.
    \item Understand the principles of agentic AI and agentic loops.
    \item Differentiate between reactive and proactive agent behavior.
    \item Describe continual learning challenges in agentic systems.
    \item Critically evaluate frameworks like Auto-GPT and BabyAGI.
    \item Reflect on safety, control, and ethical issues in agentic AI.
\end{itemize}
\end{frame}


\section{Vision-Language Models}
\begin{frame}{}
    \LARGE Vision-Language Models
\end{frame}

\subsection{CLIP: Contrastive Language-Image Pretraining}
\begin{frame}[allowframebreaks]{CLIP: Contrastive Language-Image Pretraining}
    \begin{itemize}
        \setlength{\itemsep}{1em}
        \item Developed by OpenAI (Radford et al., 2021)
        \item Trained on 400M (image, text) pairs
        \item Learns a joint embedding space using contrastive loss
        \item Image and text encoders are trained to match correct pairs
    \end{itemize}
    \vspace{0.5em}
    \textbf{Use cases:}
    \begin{itemize}
        \item Zero-shot classification
        \item Semantic similarity
        \item Prompt-based image querying
    \end{itemize}
\framebreak
    \textbf{Architecture:}
    \begin{itemize}
        \setlength{\itemsep}{1em}
        \item ResNet / ViT for images
        \item Transformer for text
        \item Cosine similarity loss
    \end{itemize}
    \vspace{0.5em}
    \textbf{Paper:} \href{https://arxiv.org/abs/2103.00020}{CLIP: Learning Transferable Visual Models From Natural Language Supervision}
\end{frame}


\subsection{VisualBERT}
\begin{frame}[allowframebreaks]{VisualBERT}
    \begin{itemize}
        \vspace{-0.5em}
        \item Joint Transformer-based model: processes both image regions and text.
        \item \textbf{Early fusion:} image embeddings are added as input tokens to the Transformer.
        \item \textbf{Pretraining objectives:}
        \begin{itemize}
            \vspace{-0.5em}
            \item Masked Language Modeling (MLM)
            \item Next Sentence Prediction (NSP)
            \item Image-Text Alignment
        \end{itemize}
    \end{itemize}
    \textbf{Applications:}
    \begin{itemize}
        \vspace{-0.5em}
        \item Visual Question Answering (VQA)
        \item Image captioning
        \item Visual commonsense reasoning
    \end{itemize}
    \textbf{Paper:} \href{https://arxiv.org/abs/1908.03557}{VisualBERT: A Simple and Performant Baseline for Vision and Language}
\end{frame}


\subsection{FLAVA: Fusion of Language And Vision Architecture}
\begin{frame}[allowframebreaks]{FLAVA (Facebook AI)}
    \begin{itemize}
        \item FLAVA = \textbf{Fusion of Language And Vision Architecture}
        \item Handles both unimodal and multimodal tasks
        \item \textbf{Three towers:} vision, language, multimodal
        \item Self-supervised objectives: masked modeling, contrastive learning
    \end{itemize}
    \vspace{0.5em}
    \textbf{Capabilities:}
    \begin{itemize}
        \item Text classification
        \item Image classification
        \item Visual Question Answering (VQA)
        \item Image captioning
    \end{itemize}
    \vspace{0.5em}
    \textbf{Paper:} \href{https://arxiv.org/abs/2112.04482}{FLAVA: A Foundational Language And Vision Alignment Model}
\end{frame}


\section{Multimodal Architectures}
\begin{frame}{}
    \LARGE Multimodal Architectures
\end{frame}


\subsection{Categories of Multimodal Architectures}
\begin{frame}[allowframebreaks]{Categories of Multimodal Architectures}
    \begin{itemize}
        \item \textbf{Early Fusion:} Combine raw modalities at the input level.\\
        \textit{Example: VisualBERT adds image embeddings as tokens to the Transformer.}
        \item \textbf{Late Fusion:} Process each modality separately, then merge representations.\\
        \textit{Example: CLIP encodes images and text independently, then compares embeddings.}
        \item \textbf{Hybrid / Cross-modal Attention:} Enable interactions between modalities at multiple stages using attention mechanisms.
    \end{itemize}
    \textbf{\faWrench\hspace{0.5em}Design Considerations:}
    \begin{itemize}
        \item \textbf{Alignment:} How to align representations across modalities.
        \item \textbf{Modality Dropout:} Handling missing or noisy modalities during training or inference.
        \item \textbf{Shared vs. Modality-Specific Layers:} Deciding which layers are shared and which are unique to each modality.
    \end{itemize}
\end{frame}


\subsection{Challenges in Multimodal Integration}
\begin{frame}[allowframebreaks]{Challenges in Multimodal Integration}
    \begin{itemize}
        \vspace{-0.5em}
        \item \textbf{Representation mismatch:} Different modalities (e.g., images, text, audio) have distinct statistical properties and structures, making joint representation learning challenging.
        \item \textbf{Missing modalities:} Real-world data may lack one or more modalities at training or inference time, requiring robust models that can handle incomplete inputs.
        \item \textbf{Synchronization and timing:} Temporal alignment is critical when modalities are sequential (e.g., video and audio), and misalignment can degrade performance.
        \item \textbf{Data scarcity in aligned datasets:} Large-scale, high-quality datasets with well-aligned multimodal pairs are rare, limiting supervised learning approaches.
        \item \textbf{Evaluation metrics:} Choosing appropriate metrics is difficult—should models be evaluated on task-specific performance or their ability to generalize across modalities?
    \end{itemize}
\end{frame}


\section{Agentic AI}
\begin{frame}{}
    \LARGE Agentic AI
\end{frame}

\subsection{What is Agentic AI?}
\begin{frame}{What is Agentic AI?}
    \begin{itemize}
        \item AI systems that \textbf{perceive}, \textbf{reason}, \textbf{plan}, and \textbf{act} to fulfill goals.
        \item \textbf{Agent loop:} Observe $\rightarrow$ Plan $\rightarrow$ Act $\rightarrow$ Reflect $\rightarrow$ Learn
        \item Inspired by cognitive science, robotics, and autonomous agents.
    \end{itemize}
    \vspace{1em}
    \textbf{Core Components:}
    \begin{itemize}
        \item Memory
        \item Planning
        \item Tool-use (plugins, APIs)
        \item Feedback-driven behavior
    \end{itemize}
\end{frame}

\subsection{Reactive vs. Proactive Agents}
\begin{frame}{Reactive vs. Proactive Agents}
    \begin{table}[]
        \centering
        \renewcommand{\arraystretch}{1.8}
        \begin{tabular}{lcc}
            \hline
            \textbf{Feature} & \textbf{Reactive} & \textbf{Proactive} \\
            \hline
            Response & Immediate & Goal-oriented \\
            Planning & None & Yes \\
            Learning & Event-triggered & Self-initiated \\
            Example & Chatbots & Auto-GPT, Personal Assistants \\
            \hline
        \end{tabular}
    \end{table}
    \vspace{1em}
    
    \textbf{Proactive agents} generate goals, schedule actions, and evaluate results even without user prompts.
\end{frame}

\subsection{Agent Loop}
\begin{frame}[allowframebreaks]{Agent Loop in Detail}
    \begin{itemize}
        \item \textbf{Perceive:} Multimodal input (text, vision, memory)
        \item \textbf{Interpret:} NLP/vision models
        \item \textbf{Plan:} Task decomposition (e.g., LangChain chains)
        \item \textbf{Act:} API/tool calling
        \item \textbf{Reflect \& Learn:} Refine memory or adjust goals
    \end{itemize}
    \vspace{1em}
    \textbf{Applications:}
    \begin{itemize}
        \item Personal AI assistants
        \item Research co-pilots
        \item Automated workflows
    \end{itemize}
\end{frame}


\section{Open-Source Agentic Frameworks}
\begin{frame}{}
    \LARGE Open-Source Agentic Frameworks
\end{frame}


\subsection{Auto-GPT}
\begin{frame}[allowframebreaks]{Auto-GPT}
    \begin{itemize}
        \item \textbf{Open-source project} that chains GPT calls to build autonomous agents.
        \item Uses \textbf{long-term memory}, file storage, and self-generated tasks.
        \item \textbf{Requirements:}
        \begin{itemize}
            \item Task input
            \item Internet / API access
            \item Feedback loop
        \end{itemize}
    \end{itemize}
    \vspace{0.5em}
    \textbf{GitHub:} \href{https://github.com/Torantulino/Auto-GPT}{https://github.com/Torantulino/Auto-GPT}
\end{frame}


\subsection{BabyAGI}
\begin{frame}[allowframebreaks]{BabyAGI}
    \begin{itemize}
        \setlength{\itemsep}{1em}
        \item \textbf{Lightweight Python agent loop}
        \item Generates tasks from high-level objectives
        \item Runs tasks, stores results, and generates new tasks iteratively
        \item Emphasizes simplicity and minimalism
    \end{itemize}
    \vspace{0.5em}
    \textbf{GitHub:} \href{https://github.com/yoheinakajima/babyagi}{https://github.com/yoheinakajima/babyagi}
\end{frame}


\subsection{Other Notable Frameworks}
\begin{frame}[allowframebreaks]{Other Notable Frameworks}
    \begin{itemize}
        \setlength{\itemsep}{1em}
        \item \textbf{LangChain:} Framework for chaining LLMs with tools, APIs, and memory. Enables complex workflows and agentic behavior.
        \item \textbf{CrewAI:} Multi-agent coordination platform for collaborative task execution.
        \item \textbf{SuperAGI:} Production-grade agent platform supporting extensibility, monitoring, and deployment.
        \item \textbf{OpenDevin:} Developer-oriented agentic IDE for automating software engineering tasks.
    \end{itemize}
\end{frame}


\section{Continual Learning in Agents}
\begin{frame}{}
    \LARGE Continual Learning in Agents
\end{frame}


\subsection{Why Continual Learning?}
\begin{frame}{Why Continual Learning?}
    \begin{itemize}
        \item Agentic AI should evolve with time:
        \begin{itemize}
            \item Learn from feedback
            \item Adapt to new tasks
            \item Retain useful knowledge
        \end{itemize}
    \end{itemize}
    \vspace{1em}
    \textbf{\faGlobe\hspace{0.5em}Real-world scenarios:}
    \begin{itemize}
        \item Personalization
        \item Environment drift
        \item Avoiding forgetting
    \end{itemize}
\end{frame}


\subsection{Challenges in Continual Learning}
\begin{frame}[allowframebreaks]{Challenges in Continual Learning}
    \begin{itemize}
        \item \textbf{Catastrophic forgetting:} Agents may lose previously acquired knowledge when learning new tasks.
        \item \textbf{Stability-plasticity dilemma:} Balancing the ability to learn new information (plasticity) with retaining old knowledge (stability).
        \item \textbf{Modality drift:} Changes or shifts in the types or distributions of input modalities over time.
        \item \textbf{Scaling memory:} Managing both semantic (general knowledge) and episodic (specific experiences) memory as agents interact with the world.
    \end{itemize}
    \vspace{1em}
    \textbf{\faWrench\hspace{0.5em}Solutions:}
    \begin{itemize}
        \item Replay memory (experience replay)
        \item Progressive networks
        \item Elastic weight consolidation (EWC)
    \end{itemize}
\end{frame}


\section{Safety, Ethics & Control}
\begin{frame}{}
    \LARGE Safety, Ethics & Control
\end{frame}


\subsection{Safety in Agentic Systems}
\begin{frame}[allowframebreaks]{Safety in Agentic Systems}
    \begin{itemize}
        \item \textbf{Risks:}
        \begin{itemize}
            \item Autonomous actions without human oversight
            \item Errors in tool or API use
            \item Prompt injection or goal hijacking
            \item Recursive self-improvement without constraints
        \end{itemize}
    \end{itemize}
    \vspace{1em}
    \textbf{\faEye\hspace{0.5em}Mitigations:}
    \begin{itemize}
        \item Approval-based execution (human-in-the-loop)
        \item Guardrails and policy enforcement
        \item Explainability (XAI) for agent decisions
    \end{itemize}
\end{frame}


\subsection{Ethics and Control}
\begin{frame}[allowframebreaks]{Ethics and Control}
    \begin{itemize}
        \item \textbf{Alignment:} Are agents acting in users’ interests?
        \item \textbf{Accountability:} Who is responsible for AI actions?
        \item \textbf{Control:} Can humans override or audit decisions?
    \end{itemize}
    \vspace{1em}
    \textbf{\faLightbulbO\hspace{0.5em}Key Areas:}
    \begin{itemize}
        \item Fairness in multimodal data
        \item Bias amplification (vision + language)
        \item Transparency in decision-making
    \end{itemize}
\end{frame}


\section{Future Directions}
\begin{frame}{}
    \LARGE Future Directions
\end{frame}


\subsection{Open Problems and Research Directions}
\begin{frame}[allowframebreaks]{Open Problems and Research Directions}
    \begin{itemize}
        \item \textbf{Multimodal grounding in real-world environments:} How can agents robustly connect language, vision, and action in dynamic, noisy settings?
        \item \textbf{World model integration:} Combining temporal and spatial memory for persistent, context-aware reasoning.
        \item \textbf{Agent collaboration:} Enabling effective communication and coordination in multi-agent systems.
        \item \textbf{Scaling memory-efficient agents:} Designing architectures that scale to long-term, large-scale memory without prohibitive costs.
        \item \textbf{Robust reward models:} Developing reliable reward and feedback mechanisms for open-ended, real-world tasks.
        \item \textbf{Meta-cognition:} Building agents that can introspect, self-monitor, and adapt their own reasoning processes.
    \end{itemize}
\end{frame}


\section{Summary}
\begin{frame}{Summary}
    \begin{itemize}
        \setlength{\itemsep}{1em}
        \item Multimodal models (CLIP, VisualBERT, FLAVA) integrate perception and language for richer AI understanding.
        \item Agentic AI enables proactive, goal-driven, and autonomous behavior.
        \item Open-source frameworks (Auto-GPT, BabyAGI) are accelerating research and applications.
        \item Ethical, safe, and explainable deployment is essential for real-world impact.
        \item The future of AI lies in continual learning and intelligent autonomy.
    \end{itemize}
\end{frame}


\section*{References}
\begin{frame}[allowframebreaks]{References}
    \scriptsize
    \begin{thebibliography}{99}
        \bibitem{radford2021clip}
        Radford, A., Kim, J. W., Hallacy, C., et al. (2021).
        \newblock CLIP: Learning Transferable Visual Models From Natural Language Supervision.
        \newblock \textit{arXiv:2103.00020}.

        \bibitem{li2019visualbert}
        Li, L. H., Yatskar, M., Yin, D., Hsieh, C. J., & Chang, K. W. (2019).
        \newblock VisualBERT: A Simple and Performant Baseline for Vision and Language.
        \newblock \textit{arXiv:1908.03557}.

        \bibitem{singh2021flava}
        Singh, A., Goyal, N., Goswami, V., et al. (2021).
        \newblock FLAVA: A Foundational Language And Vision Alignment Model.
        \newblock \textit{arXiv:2112.04482}.

        \bibitem{yao2023react}
        Yao, S., Zhao, J., Yu, D., et al. (2023).
        \newblock ReAct: Synergizing Reasoning and Acting in Language Models.
        \newblock \textit{arXiv:2210.03629}.

        \bibitem{nakajima2023babyagi}
        Nakajima, Y. (2023).
        \newblock BabyAGI.
        \newblock \url{https://github.com/yoheinakajima/babyagi}

        \bibitem{torantulino2023autogpt}
        Torantulino (2023).
        \newblock Auto-GPT.
        \newblock \url{https://github.com/Torantulino/Auto-GPT}

        \bibitem{ahn2022saycan}
        Ahn, M., Brohan, A., Chebotar, Y., et al. (2022).
        \newblock Do As I Can, Not As I Say: Grounding Language in Robotic Affordances.
        \newblock \textit{arXiv:2204.01691}.

        \bibitem{openai2023agentsim}
        OpenAI Blog.
        \newblock Agent Simulations and Superalignment.
        \newblock \url{https://openai.com/blog/}
    \end{thebibliography}
\end{frame}