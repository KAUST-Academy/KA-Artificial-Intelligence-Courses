\begin{frame}[allowframebreaks]{Denoising}
    \textbf{Key Idea:}
    \begin{itemize}
        \item Train the autoencoder to reconstruct the original, clean input from a corrupted (noisy) version.
        \item The encoder learns robust features that capture the underlying structure of the data, ignoring noise.
    \end{itemize}

    \vspace{1em}
    \textbf{Training Process:}
    \begin{enumerate}
        \item Add noise to the input data (e.g., Gaussian noise, salt-and-pepper noise).
        \item Feed the noisy input to the autoencoder.
        \item Compute the loss between the output and the original clean input.
        \item Update the network weights to minimize this loss.
    \end{enumerate}

    \framebreak

    \textbf{Use Cases:}
    \begin{itemize}
        \item Image denoising: Removing noise from photographs or medical images.
        \item Speech enhancement: Improving audio quality by filtering out background noise.
        \item Preprocessing step: Providing cleaner data for downstream tasks such as classification or segmentation.
    \end{itemize}

    \framebreak

    \begin{figure}
        \centering
        \fetchconvertimage{https://wikidocs.net/images/page/193827/Autoencoders-to-Image-Denoising.png}{images/autoencoders/denoising.png}{width=\textwidth,keepaspectratio}
        \caption*{Example of denoising using an autoencoder.}
    \end{figure}
\end{frame}