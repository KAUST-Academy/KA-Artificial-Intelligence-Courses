\begin{frame}[allowframebreaks]{}
    \begin{figure}
        \centering
        \fetchconvertimage{https://framerusercontent.com/images/r0i6jmANAR5bsgzvqm0BJG9aas.png}{images/ssl/architecture.png}{height=0.9\textheight,width=1\textwidth,keepaspectratio}
    \end{figure}
\end{frame}

\begin{frame}{Motivation for Self-Supervised Learning}
    \begin{enumerate}
        \setlength{\itemsep}{-0.5em}
        \item \textbf{Imagine you have tons of data, but very few labels.}\\
        Annotating data is expensive and slow, but unlabeled data is everywhere!\\[0.5em]
        \item<2-> \textbf{What if you could learn useful features from all that unlabeled data?}\\
        Self-supervised learning lets us pre-train models to extract general, reusable representations, reducing the need for labeled data later.\\[0.5em]
        \item<3-> \textbf{Think beyond images:}\\
        These techniques work not just for vision, but also for language, audio, and time-series data, enabling unified approaches across domains.\\[0.5em]
        \item<4-> \textbf{Inspired by how humans learn:}\\
        We learn patterns from the world around us without explicit labels—self-supervised learning mimics this natural process.\\[0.5em]
        \item<5-> \textbf{A journey of methods:}\\
        The field has evolved from generative models (like VAEs and GANs), to contrastive and predictive coding, and now to hybrid approaches combining reconstruction and discrimination.
    \end{enumerate}
    % \begin{itemize}
    %     \item \textbf{Data Label Bottleneck}
    %     \begin{itemize}
    %         \item Annotating large datasets is time-consuming and expensive.
    %         \item Vast amounts of unlabeled data (images, audio, text) are readily available but often underutilized.
    %     \end{itemize}
    %     \item \textbf{Reusable Representations}
    %     \begin{itemize}
    %         \item Pre-training on unlabeled data enables learning of general, transferable features.
    %         \item Reduces the need for extensive task-specific labeled data.
    %     \end{itemize}
    %     \item \textbf{Cross-Modal and Multi-Task Learning}
    %     \begin{itemize}
    %         \item Self-supervised frameworks can be applied across different domains: vision, language, audio, and time-series.
    %         \item Facilitates unified approaches to diverse tasks.
    %     \end{itemize}
    %     \item \textbf{Biological Inspiration}
    %     \begin{itemize}
    %         \item Humans learn patterns and structure from sensory input without explicit labels.
    %         \item Self-supervised learning mimics this natural learning process.
    %     \end{itemize}
    %     \item \textbf{Evolution of Learning Paradigms}
    %     \begin{itemize}
    %         \item Progression from generative models (e.g., VAEs, GANs) to contrastive and predictive coding methods.
    %         \item Modern self-supervised methods combine reconstructive and discriminative objectives.
    %     \end{itemize}
    % \end{itemize}
\end{frame}