\begin{frame}{How Does ML Work?}
    \begin{itemize}
        \item Most of the ML systems consist of three main components:
    \end{itemize}

    \begin{itemize}
        \item \textbf{Hypothesis (Model)}: The function that approximates the target.
        \begin{itemize}
            \item E.g. Linear Regression, Logistic Regression, SVM, Decision Trees, NN, ...
        \end{itemize}

        \item \textbf{Optimizer}: The mechanism for improving predictions of our model.

        \item \textbf{Loss Function}: The measure of how wrong the predictions are.
    \end{itemize}
\end{frame}


\begin{frame}{How Does ML Work?}
    \begin{itemize}
        \item How are they related to each other? \includegraphics[height=1em]{images/emojis/thinking-face.png}
    \end{itemize}
\end{frame}

\begin{frame}{How Does ML Work?}
    \begin{itemize}
        \item We firstly define our task (classification/regression) then choose an appropriate \underline{\textbf{model}}.
        \item We will use an \underline{\textbf{optimization method}} to minimize the \underline{\textbf{loss function}}.
        \item Reached a minima?
        \begin{itemize}
            \item Model is making the least possible number of mistakes.
            \item \underline{\textbf{Model trained}}~\includegraphics[height=1em]{images/emojis/party-popper.png}.
        \end{itemize}
    \end{itemize}
\end{frame}


