\begin{frame}{}
	\LARGE VAE: \textbf{Reparameterization Trick (Pathwise Derivative Estimator)}
\end{frame}

\begin{frame}[allowframebreaks]{Reparameterization Trick}
\textbf{Problem:} Cannot backpropagate through stochastic sampling.\\[1em]

\textbf{Solution:} Reparameterize $z$ as:
\[
z = \mu + \sigma \cdot \epsilon, \quad \epsilon \sim \mathcal{N}(0, I)
\]

\textbf{Benefit:} Enables gradient-based optimization by making the sampling operation differentiable.

\framebreak

\begin{figure}
	\centering
	\includegraphics[height=0.8\textheight, width=\textwidth, keepaspectratio]{images/vae/reparam_trick.png}
	\caption*{Reparameterization trick to make back propagation possible}
\end{figure}

\framebreak

\begin{figure}
	\centering
	\includegraphics[height=0.8\textheight, width=\textwidth, keepaspectratio]{images/vae/reparam_trick_network.png}
	\caption*{Variational Autoencoder with reparameterization trick}
\end{figure}
\end{frame}

\begin{frame}[allowframebreaks]{Pathwise Derivative - Toy Problem}
\begin{figure}
		\centering
		\includegraphics[height=0.9\textheight, width=\textwidth, keepaspectratio]{images/vae/toy-problem-1.png}
\end{figure}

\framebreak

\begin{figure}
		\centering
		\includegraphics[height=0.8\textheight, width=\textwidth, keepaspectratio]{images/vae/toy-problem-2.png}
\end{figure}

\framebreak

\begin{figure}
		\centering
		\includegraphics[height=0.8\textheight, width=\textwidth, keepaspectratio]{images/vae/toy-problem-3.png}
\end{figure}

\framebreak

\begin{figure}
		\centering
		\includegraphics[height=0.8\textheight, width=\textwidth, keepaspectratio]{images/vae/toy-problem-4.png}
\end{figure}
\end{frame}