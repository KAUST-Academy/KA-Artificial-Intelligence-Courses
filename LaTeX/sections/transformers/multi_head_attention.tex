\section{Multi-Head Attention}
\begin{frame}
  \frametitle{Multi-Head Attention}
  \begin{itemize}
    \item Multi-head attention allows the model to jointly attend to information from different representation subspaces at different positions.
    \item It consists of multiple attention heads, each with its own set of parameters.
    \item The outputs of all heads are concatenated and linearly transformed.
  \end{itemize}
  \textbf{Formula:}
  \[
  \text{MultiHead}(Q, K, V) = \text{Concat}\left(\text{head}_1, \ldots, \text{head}_h\right) W^O
  \]
  where each head is defined as:
  \[
  \text{head}_i = \text{Attention}(Q W_i^Q, K W_i^K, V W_i^V)
  \]
  \begin{itemize}
    \item $W_i^Q$, $W_i^K$, and $W_i^V$ are learned projection matrices for each head.
    \item $W^O$ is the output projection matrix that combines the outputs of all heads.
  \end{itemize}
\end{frame}

\begin{frame}
    \frametitle{Benefits of Multi-Head Attention}
    \begin{itemize}
        \item Enables the model to jointly attend to information from different representation subspaces at different positions.
        \item Captures various linguistic and syntactic patterns by using multiple attention heads.
        \item Improves the model's ability to represent complex relationships in the data.
        \item Provides richer and more diverse feature representations.
    \end{itemize}
\end{frame}