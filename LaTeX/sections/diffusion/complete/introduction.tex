\section{Introduction}
\begin{frame}{}
    \LARGE Diffusion Models: \textbf{Introduction}
\end{frame}

\begin{frame}{Introduction}

Diffusion models are a class of probabilistic generative models that learn to reverse a gradual noising process applied to data. They have gained popularity due to their ability to generate samples with high fidelity and diversity.

\begin{itemize}
    \item<2-> \textbf{Inspiration:} Rooted in concepts from thermodynamics and stochastic processes.
    \item<3-> \textbf{Key Idea:} 
    \begin{itemize}
        \item \textit{Forward process:} Progressively corrupt data by adding Gaussian noise.
        \item \textit{Reverse process:} Learn to recover the original data by reversing the noising process.
    \end{itemize}
    \item<4-> \textbf{Popularity:} Known for producing high-quality and diverse generated samples.
\end{itemize}
\end{frame}

\begin{frame}{Quick Overview}
    \begin{itemize}
        \item \textbf{Forward:} Clean $\rightarrow$ Noise (additive Gaussian steps)
        \item \textbf{Reverse:} Learn \textbf{U-Net} to remove noise step by step
        \item \textbf{Training:} Model predicts the noise added
        \item \textbf{Sampling:} Roll back from pure noise to a data sample
    \end{itemize}
\end{frame}