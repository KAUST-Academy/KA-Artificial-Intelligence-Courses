% Table of Contents
\begin{frame}[t, allowframebreaks]{Table of Contents}
\begin{enumerate}
    \item Motivation
    \item Learning Outcomes
    \item Concept
    \item Architecture
    \begin{enumerate}
        \item Encoder
        \item Decoder
        \item Bottleneck
    \end{enumerate}
    \item As Generative Models
    \item Applications
        \begin{enumerate}
            \item Dimensionality Reduction
            \item Super-Resolution
            \item Colorization
            \item Anomaly Detection
            \item Denoising
        \end{enumerate}
    \item Challenges
    \item References
\end{enumerate}
\end{frame}


% \begin{frame}{Introduction}
%     Autoencoders are a type of artificial neural network used for learning efficient codings of input data in an unsupervised manner. The goal is to learn a compressed (encoded) representation of the input and then reconstruct the original input from this compressed version.
%     \begin{itemize}
%         \item \textbf{Developed originally in the 1980s}, autoencoders have gained renewed popularity due to advances in deep learning and hardware.
%         \item Often used in \textbf{dimensionality reduction, denoising, anomaly detection, and generative modeling}.
%     \end{itemize}
% \end{frame}