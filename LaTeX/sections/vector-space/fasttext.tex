\section{fastText – Subword Embeddings}
\begin{frame}{}
    \LARGE fastText – Subword Embeddings
\end{frame}

\begin{frame}[allowframebreaks]{fastText – Key Ideas}
    \begin{itemize}
        \item Developed by Facebook AI (2016)
        \item Builds on Word2Vec by incorporating character n-grams
        \item Useful for morphologically rich languages and rare words
        \item Better handling of OOV (out-of-vocabulary) words
    \end{itemize}
\framebreak
    \begin{figure}
        \centering
        \fetchconvertimage{https://miro.medium.com/v2/resize:fit:1200/1*AgrrRZ9DpUVb3srTWs0gzA.png}{images/vector-space/fasttext-1.png}{width=\textwidth,height=0.9\textheight,keepaspectratio}
    \end{figure}
\framebreak
    \begin{figure}
        \centering
        \fetchconvertimage{https://pub.mdpi-res.com/information/information-10-00161/article_deploy/html/images/information-10-00161-g001.png?1571471085}{images/vector-space/fasttext-2.png}{width=\textwidth,height=0.8\textheight,keepaspectratio}
        \caption*{FastText-Based Intent Detection for Inflected Languages}
    \end{figure}
\end{frame}

\begin{frame}{fastText – Example}
    \begin{itemize}
        \item Word: \texttt{playing}
        \item Character n-grams (n=3): \texttt{["pla", "lay", "ayi", "yin", "ing"]}
        \item Word vector = sum of n-gram vectors
    \end{itemize}
\end{frame}

\begin{frame}{Comparison: Word2Vec vs. GloVe vs. fastText}
    \begin{table}[h!]
        \centering
        \renewcommand{\arraystretch}{1.8}
        \begin{tabular}{lccc}
            \toprule
            \textbf{Feature} & \textbf{Word2Vec} & \textbf{GloVe} & \textbf{fastText} \\
            \midrule
            Local Context   & \checkmark & -- & \checkmark \\
            Global Info     & --         & \checkmark & \checkmark (partial) \\
            Subword Info    & --         & -- & \checkmark \\
            Handles OOV     & --         & -- & \checkmark \\
            \bottomrule
        \end{tabular}
    \end{table}
\end{frame}